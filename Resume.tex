%%%%%%%%%%%%%%%%%%%%%%%%%%%%%%%%%%%%%%%%%%%%%%%%%%%%%%%%%%%%%%%%%%%%%%%%%%%%%%%%%%%%%%%%%%%%%
%%									Résumé									%
%%%%%%%%%%%%%%%%%%%%%%%%%%%%%%%%%%%%%%%%%%%%%%%%%%%%%%%%%%%%%%%%%%%%%%%%%%%%%%%%%%%%%%%%%%%%%
\chapter*{Résumé}
%\pagenumbering{arabic}
 
\addstarredchapter{Résumé}
%	\epigraph{\og Une super citation, si vous êtes inspiré d'entrée de jeu. \fg{}}{D. Depriester}
\markboth{RESUME}{}
%%%%%%%%%%%%%%%%%%%%%%%%%%%%%%%%%%%%%%%%%%%%%%%%%%%%%%%%%%

%%%%%%%%%%%%%%%%%%%%%%%%%%%%%%%%%%%%%%%%%%%%%%%%%%%%%%%%%%%%%%%%%%%%%%%%%%%%%%%%%%%%%%%%%%%%%
% Début du chapitre

Cette thèse a pour objet d\rq{}étude les méthodes numériques d\rq{}approximation de la densité de probabilité associée à des variables aléatoires admettant des distributions composées. Ces variables aléatoires sont couramment utilisées en actuariat pour modéliser le risque supporté par un portefeuille de contrats. En théorie de la ruine, la probabilité de ruine ultime dans le modèle de Poisson composé est égale à la fonction de survie d\rq{}une distribution géométrique composée. La méthode numérique proposée consiste en une projection orthogonale de la densité sur une base de polynômes orthogonaux. Ces polynômes sont orthogonaux par rapport à une mesure de probabilité de référence appartenant aux Familles Exponentielles Naturelles Quadratiques. La méthode d\rq{}approximation polynomiale est comparée à d\rq{}autres méthodes d\rq{}approximation de la densité basées sur les moments et la transformée de Laplace de la distribution. L\rq{}extension de la méthode en dimension supérieure à $1$ est présentée, ainsi que l\rq{}obtention d\rq{}un estimateur de la densité à partir de la formule d\rq{}approximation. Cette thèse comprend aussi la description d\rq{}une méthode d\rq{}agrégation adaptée aux portefeuilles de contrats d\rq{}assurance vie de type épargne individuelle. La procédure d\rq{}agrégation conduit à la construction de \textit{model points} pour permettre l\rq{}évaluation des provisions \textit{best estimate} dans des temps raisonnables et conformément à la directive européenne Solvabilité II.\\ 

\textbf{\textit{Mots-clé:}} Distributions composées, théorie de la ruine, Familles Exponentielles Naturelles Quadratiques, polynômes orthogonaux, méthodes numériques d\rq{}approximation, Solvabilité II, provision \textit{best estimate}, \textit{model points}.