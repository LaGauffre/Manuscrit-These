%%%%%%%%%%%%%%%%%%%%%%%%%%%%%%%%%%%%%%%%%%%%%%%%%%%%%%%%%%%%%%%%%%%%%%%%%%%%%%%%%%%%%%%%%%%%%
%%									Remerciements										%
%%%%%%%%%%%%%%%%%%%%%%%%%%%%%%%%%%%%%%%%%%%%%%%%%%%%%%%%%%%%%%%%%%%%%%%%%%%%%%%%%%%%%%%%%%%%%
\chapter*{Remerciements}
\pagenumbering{arabic}
	\citationChap{
	A Tata Annette.
	}{}
 
\addstarredchapter{Remerciements}
%	\epigraph{\og Une super citation, si vous êtes inspiré d'entrée de jeu. \fg{}}{D. Depriester}
\markboth{REMERCIEMENTS}{}

%%%%%%%%%%%%%%%%%%%%%%%%%%%%%%%%%%%%%%%%%%%%%%%%%%%%%%%%%%%%%%%%%%%%%%%%%%%%%%%%%%%%%%%%%%%%%



% Début du chapitre

Je tiens à remercier Claude Lefevre et Patrice Bertail pour avoir accepté de rapporter cette thèse. Je remercie Dominique Henriet pour avoir accepté de faire partie du jury. J'en suis très honoré.\\

Je remercie bien entendu mes deux directeurs de thèse, Denys Pommeret et Stéphane Loisel. Sans vous ce travail n'aurait jamais existé. Merci pour le temps que vous avez pu me consacrer, les conseils toujours pertinents que vous m'avez prodigués et le goût de la recherche que vous avez su m'insuffler. Je garderai en mémoire les bons moments passés en marge des conférences ou encore les après-midi de travail à l'ISFA post restaurant, toujours très fructueuses...\\

J'adresse un grand merci à Xavier Guerrault, mon "responsable-entreprise" chez AXA, qui a su se montrer très patient avec moi. Je sais que je peux parfois êre difficile à manager. J'ai beaucoup apprécié travailler avec toi, et je suis très triste à l'idée que ça s'arrête. Tu m'as fait confiance jusqu'au bout et je sais que sans toi cette thèse CIFRE n'aurait jamais vu le jour. Tu es un exemple à suivre professionnellement, peut-être as-tu raison un peu trop souvent ce qui est un peu énervant! J'en profite pour remercier Mohammed Baccouche, pour le soutien qu'il a apporté au projet \textit{Model Point}, et Isabelle Jubelin en charge de la gestion du fonds AXA pour la recherche qui m'a permis de terminer cette thèse dans les meilleures conditions.\\

J'ai une pensée pour mes parents et ma famille, qui m'ont toujours soutenu, et supporté malgré mes sauts d'humeur et mes phases de "débranchage" lors de mes retours à Lorient. Je remercie mes parents d'avoir fait le maximum pour que ma soutenance se déroule dans les meilleurs conditions, je ne sais pas comment j'aurais fait sans vous. Je remercie Louis-Marie, Nathalie, Corentin, et Amélie d'avoir fait le déplacement jusque dans cette sulfureuse cité phocéenne, mes deux grands-parents dont la présence est toujours réconfortante, enfin mes deux frangins Simon et Vincent qui ont pris leur part de responsabilité dans l'organisation de cette merveilleuse soirée de soutenance!
\\

Je remercie tous les gens que j'ai eu la chance de rencontrer au sein d'AXA. J'ai passé de très bons moments avec vous pendant les pauses café, les repas de midi, mais aussi les foots en salle, les barbeucs, les plages, les restos avec piscine et les soirées en tout genre! Merci à Jennifer, Julie, Céline, Modou, Amin et David de l'équipe épargne individuelle.Merci à Patricio, Tom et JP de la retraite collective. Merci à Vince, Badr, Mathieu, Marilyne, Marion, Pierre-Etienne et Emmanuele dans l'équipe prévoyance individuelle. Merci à Anais, Loic, Sophie, Maud V, Maud M, Olive, Christine, et Thérèse de l'équipe des comptes IARD. Merci à Erwan qui aurait vraiment toute sa place dans cette équipe IARD. Merci à Alban, Fida, John, Laure, Anwar, Hugo, Leslie, Delphine, Corinne, et Carelle dans l'équipe RMMV. Merci à Raph et à Bénédicte de la formation, ainsi qu'au Poppy Chap des Ressources Humaines. Des remerciements spéciaux vont à Viken pour tous les BBQ que tu as organisés, ta bonne humeur, les discussions toujours passionantes et conflictuelles (les fameux "débats-minute" des repas de midi), et l'animation générale que tu parviens à générer à AXA Marseille.  Je remercie spécialement Romain Silvestri car tu m'as hébergé lors des semaines d'exam à Lyon, et incrusté aux soirées de fin d'exam, Good times!
\\

Je veux remercier toutes les personnes rencontrées au laboratoire de mathématiques à Luminy, même si mon intégration a été difficile car je ne fais pas des "vraies" maths. Je garde énormément de très bons souvenirs des foots à Luminy, des sémi-bières, de la rando dans les calanques, des soirées au barberousse, des soirées chez Joël, des soirées Tonneaux-Bagel, des nombreuses parties de belotte contrée, et bien sûr l'organisation de cet immense évènement qu'est le Pi Day! Merci à mes amis thésards Emilie, J-B ou Jean Ba, Kimo, Marc M, Marc B, Eugenia, Fabio, Jordi, Fra, Matteo, Lionel, Paolo I, Paolo II, Marcello, Florent, Guillaume, et Sarah l'americaine. Merci aux membres de la team stat de l'I2M Mohamed, Badhi et Laurence. Merci à Jean-Bruno, l'apple genius du labo, toujours la en cas de soucis informatique! J'adresse des remerciements spéciaux à Joël Cohen qui a pris le temps de relire ma thèse pour traquer les typos! Les discussions que l'on a pu avoir ont toujours été riches en enseignement, et montrent qu'une collaboration entre un matheux appliqué et un matheux fondamental est possible (même si dans le cas présent cela n'a marché que dans un sens, ce n'est pas comme si je pouvais aider à résoudre des problèmes dans les groupes p-adiques tordus...)! J'adresse aussi un gigantesque merci à Anna Iezzi pour toute la bonne humeur, et le dynamisme qui la caractérisent. Tu es la Chief Happiness Officer du labo. La vie sans toi y serait bien terne. Je te remercie car tu t'es occupée de moi lors de mes week-ends passer à Luminy (surtout en fin de rédaction). Tu m'as soutenu moralement et nutritivement à base de pâtes au moment où j'en avais le plus besoin. Tu bénéficies d'une infinité de Jokers.\\

Je remercie mes amis de Lorient, Romain, Jerem, Jo, Raph, et Gwenn bien entendu. Vous avez toujours su répondre présent lors de mes retours en Bretagne pour aller se pinter un brun dans les bars, et les boîtes de Lorient, chiller à la plage, au sonisphere et à Nantes (epic nights out). J'en viens naturellement à remercier mes amis du lycée, Julien, Awen, Lucas, Romain, Carole, Tonton, Greg, Julie, Mathieu, Solène, Laura, Manon, Sarah, Lucile, Camille, Maena, et Marjo. C'est vrai qu'il devient de plus en plus difficile de se réunir et qu'il faut attendre des évènements comme le nouvel an, les EVG et les mariages. J'espère que "les vacs entre potes qui depotent et Ah que ça va être la grosse teuf" deviendront une tradition qui perdurera!\\

Je remercie tous mes potes de l'ENSAI, Kik,Kub, Adam, Chlo, Tilde, Chris, Tiny, Virgule, Isa, Guigui, Picot, Lulu, Julie, Marie, Justine, Charlou, Gregou, Cyb, Agathe,Elise, Clem, et j'en oublie... Pour toutes les soirées sur Paris avec souvent hébergement à la clé, les vacs au ski, Solidays, les Week End à Marseille - WEM, et bien sur les Week end des Amis de l'Apéro - WAA! Un merci spécial à Boris mon frère de thèse, même si tu n'as pas pu venir à la soutenance je t'aime bien quand même. On aura réussi à vivre deux ou trois aventures durant ces trois ans et demi. Je citerais pèle-mèle le dry humping, Ange Marie, la ferme aux crocodiles et Billy.\\

Je remercie mes amis de Marseille, à commencer par Laurent mon ancien colocataire et partenaire du Marseille-Cassis, qui m'a introduit dans son groupe d'amis ("la gangrène"). Merci à Diane, Melanie, Magalie, Julien, et Marie. Que de bons moments passés ensemble lors des week end aux Lecques, les soirées chez Melanie Laurent, les indénombrables Brady's + Pizza, et tous les anniversaires surprise! Un merci spécial à Cédric, le quintal avec qui j'ai pu passer des soirées assez mémorables dans les bars de Marseille $\#$TournéeDesGrandsDucs, et merci de m'avoir introduit à la team des blacks shorts de la blue coast. Je remercie donc mes amis surfeurs de Medittérannée, Cédric (dit "le gros"), Nico, Pimp, Antho, Flo, Quentin (même s'il fait du body et ne surf pas en Med), et bien sûr Ludo la maquina! Merci pour toutes ces sessions sur la côte bleue, les trips camion dans les landes et au pays Basque (voire "en el pais basco"), sans oublier les surf trips au Maroc chez notre ami Abdessamad! Je remercie pour finir les collocataires que j'ai eu la chance de me frapper lors de mes débuts à Marseille, Gilles, Lisa, Valérie, et Anne-Victoria!
On était comme \textit{un petit famille}.\\

%Il est usuel de remercier sa petite amie à la fin de cette partie. Vu ma situation, j'ai décidé de remercier l'ensemble des filles que j'ai eues la chance de fréqenter (plus ou moins longtemps, avec plus ou moins de succès) lors de ces 3 ans et demi de thèse. Merci à "Ich liebe dich", la mexicaine du trolley, l'allemande tombée dans le port, la meuf du yéti, l'onco pédiatre detentrice de l'unique psaltérion pour gaucher de France peut-être même d'Europe, la meuf du mistral, la love story de 3 jours, l'assistante sociale, Asnières-sur-Seine, la respo bar, celle qui n'était pas la SPA, la serveuse saisonière d'Aussois, la rechute, l'étudiante, la RH, "J'ai rêvé de toi hier soir, et c'était très chaud", la Djihadiste, l'essai Tinder, "Tu veux vraiment prendre ce métro?", la majorette, Marie O'Malley, Ramasse miettes et pour finir l'inchopable!\\
%
En vous souhaitant une bonne lecture,\\
\\
\textit{Pierre-O}.   
   