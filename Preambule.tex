%% Copyright (C) 2014 Dorian Depriester
%% http://blog.dorian-depriester.fr
%%
%% This file may be distributed and/or modified under the conditions
%% of the LaTeX Project Public License, either version 1.3c of this
%% license or (at your option) any later version. The latest version
%% of this license is in:
%%
%%    http://www.latex-project.org/lppl.txt
%%
%% and version 1.3c or later is part of all distributions of LaTeX
%% version 2006/05/20 or later.
%%
%% This work has the LPPL maintenance status `maintained'.
%%
%% The Current Maintainer of this work is Dorian Depriester
%% <contact [at] dorian [-] depriester [dot] fr>.
%%
%% This is Preambule.tex for French PhD Thesis.



%%%%%%%%%%%%%%%%%%%%%%%%%%%%%%%%%%%%%%%%
%           Liste des packages         %
%%%%%%%%%%%%%%%%%%%%%%%%%%%%%%%%%%%%%%%%


%% Faux texte, juste pour la démo
\usepackage{blindtext}

%%%%%%%%%%%%%%%%%%%%%%%%%%%%%%%%%%%%%%%%%%%%%%%%%%%%%%%%%%%%%%%%%%%%%

%% Réglage des fontes et typo    
\usepackage[utf8]{inputenc}		% LaTeX, comprend les accents !
\usepackage[T1]{fontenc}

\usepackage[square,sort&compress,sectionbib]{natbib}		% Doit être chargé avant babel
\usepackage{chapterbib}
	\renewcommand{\bibsection}{\section{Références}}		% Met les références biblio dans un \section (au lieu de \section*)
		
\usepackage[frenchb]{babel}
\usepackage{lmodern}
\usepackage{ae,aecompl}										% Utilisation des fontes vectorielles modernes
\usepackage[upright]{fourier}



%%%%%%%%%%%%%%%%%%%%%%%%%%%%%%%%%%%%%%%%%%%%%%%%%%%%%%%%%%%%%%%%%%%%%

%% Apparence globale             
\usepackage[top=2.5cm, bottom=2cm, left=3cm, right=2.5cm,
			headheight=15pt]{geometry} 
\usepackage{fancyhdr}			% Entête et pieds de page
	\pagestyle{fancy}			% Indique que le style de la page sera justement fancy
	\lfoot[\thepage]{} %gauche du pied de page
	\cfoot{} %milieu du pied de page
	\rfoot[]{\thepage} %droite du pied de page
	\fancyhead[RO, LE] {}	
\usepackage{enumerate}
\usepackage{enumitem}
\usepackage[section]{placeins}	% Place un FloatBarrier à chaque nouvelle section
\usepackage{epigraph}
\usepackage[font={small}]{caption}
\usepackage[francais]{minitoc}		% Mini table des matières, en français
	\setcounter{minitocdepth}{2}	% Mini-toc détaillées (sections/sous-sections)
\usepackage{pdflscape}				% Permet d'utiliser des pages au format paysage

%%%%%%%%%%%%%%%%%%%%%%%%%%%%%%%%%%%%%%%%%%%%%%%%%%%%%%%%%%%%%%%%%%%%%

%% Maths                         
\usepackage{amsmath}			% Permet de taper des formules mathématiques
\usepackage{amssymb}			% Permet d'utiliser des symboles mathématiques
\usepackage{amsfonts}			% Permet d'utiliser des polices mathématiques
\usepackage{nicefrac}
\usepackage{upgreek}			% For roman (i.e. upright) lowercase Greek characters
\usepackage{multicol,multirow,vmargin,cancel,lastpage,fancyhdr,pst-all}% module de symboles math\`{e}matiques ou non
\usepackage{amsthm}

%%%%%%%%%%%%%%%%%%%%%%%%%%%%%%%%%%%%%%%%%%%%%%%%%%%%%%%%%%%%%%%%%%%%%


%% Tableaux
\usepackage{multirow}
\usepackage{booktabs}
\usepackage{colortbl}
\usepackage{tabularx}
\usepackage{multirow}
\usepackage{threeparttable}
\usepackage{etoolbox}
	\appto\TPTnoteSettings{\footnotesize}
\addto\captionsfrench{\def\tablename{{\textsc{Tableau}}}}	% Renome 'table' en 'tableau'

            
            

%%%%%%%%%%%%%%%%%%%%%%%%%%%%%%%%%%%%%%%%%%%%%%%%%%%%%%%%%%%%%%%%%%%%%
%% Graphiques                    
\usepackage{graphicx}			% Permet l'inclusion d'images
\usepackage{subcaption}
\usepackage{pdfpages}
\usepackage{rotating}
\usepackage{pgfplots}
	\usepgfplotslibrary{groupplots}
\usepackage{tikz}
	\usetikzlibrary{backgrounds,automata}
	\pgfplotsset{width=7cm,compat=1.3}
	\tikzset{every picture/.style={execute at begin picture={
   		\shorthandoff{:;!?};}
	}}
	\pgfplotsset{every linear axis/.append style={
		/pgf/number format/.cd,
		use comma,
		1000 sep={\,},
	}}
\usepackage{eso-pic}
\usepackage{import}
\usepackage{cclicenses}

%%%%%%%%%%%%%%%%%%%%%%%%%%%%%%%%%%%%%%%%%%%%%%%%%%%%%%%%%%%%%%%%%%%%%
% Biblio                        
\makeatletter
\patchcmd{\BR@backref}{\newblock}{\newblock(page~}{}{}	% Pour les back-references, affiche 'page' au lieu de 'p.'
\patchcmd{\BR@backref}{\par}{)\par}{}{}
\makeatother
	
	
%%%%%%%%%%%%%%%%%%%%%%%%%%%%%%%%%%%%%%%%%%%%%%%%%%%%%%%%%%%%%%%%%%%%%
%% Navigation dans le document   
\usepackage[pdftex,pdfborder={0 0 0},
			colorlinks=true,
			linkcolor=blue,
			citecolor=red,
			pagebackref=true,
			]{hyperref} %Créera automatiquement les liens internes au PDF


%%%%%%%%%%%%%%%%%%%%%%%%%%%%%%%%%%%%%%%%%%%%%%%%%%%%%%%%%%%%%%%%%%%%%
%% Mise en forme du texte        
\usepackage{xspace}
\usepackage[load-configurations = abbreviations]{siunitx}
	\DeclareSIUnit{\MPa}{\mega\pascal}
	\DeclareSIUnit{\micron}{\micro\meter}
	\DeclareSIUnit{\tr}{tr}
	\DeclareSIPostPower\totheM{m}
	\sisetup{
	locale = FR,
	  inter-unit-separator=$\cdot$,
	  range-phrase=~\`{a}~,     	% Utilise le tiret court pour dire "de... à"
	  range-units=single,  			% Cache l'unité sur la première borne
	  }
%\usepackage{chemist}
\usepackage[version=3]{mhchem}
\usepackage{textcomp}
\usepackage{numprint}
\usepackage{array}
\usepackage[acronym,xindy,toc,numberedsection]{glossaries}
	\newglossary[nlg]{notation}{not}{ntn}{Notation} 	% Création d'un type de glossaire 'notation'
	\makeglossaries
	\loadglsentries{Glossaire}							% Utilisation d'un fichier externe pour la définition des entrées (Glossaire.tex)
\usepackage{hyphenat}





%%%%%%%%%%%%%%%%%%%%%%%%%%%%%%%%%%%%%%%%%%%%%%%%%%%%%%%%%%%%%%%%%%%%%
%% Compilation

\usepackage{silence}
%
%% Virer les erreur dues à minitoc
\WarningFilter{minitoc(hints)}{W0023}
\WarningFilter{minitoc(hints)}{W0024}
\WarningFilter{minitoc(hints)}{W0028}
\WarningFilter{minitoc(hints)}{W0030}

 


%%%%%%%%%%%%%%%%%%%%%%%%%%%%%%%%%%%%%%%%
%           Page de garde              %
%%%%%%%%%%%%%%%%%%%%%%%%%%%%%%%%%%%%%%%%
\makeatletter
\def\@ecole{école}
\newcommand{\ecole}[1]{
  \def\@ecole{#1}
}

\def\@specialite{Spécialité}
\newcommand{\specialite}[1]{
  \def\@specialite{#1}
}

\def\@directeur{directeur}
\newcommand{\directeur}[1]{
  \def\@directeur{#1}
}

\def\@encadrant{encadrant}
\newcommand{\encadrant}[1]{
  \def\@encadrant{#1}
}

\def\@supervisor{supervisor}
\newcommand{\supervisor}[1]{
  \def\@supervisor{#1}
}
\def\@jurya{}{}{}
\newcommand{\jurya}[3]{
  \def\@jurya{#1,	& #2	& #3\\}
}
\def\@juryb{}{}{}
\newcommand{\juryb}[3]{
  \def\@juryb{#1,	& #2	& #3\\}
}
\def\@juryc{}{}{}
\newcommand{\juryc}[3]{
  \def\@juryc{#1,	& #2	& #3\\}
}
\def\@juryd{}{}{}
\newcommand{\juryd}[3]{
  \def\@juryd{#1,	& #2	& #3\\}
}
\def\@jurye{}{}{}
\newcommand{\jurye}[3]{
  \def\@jurye{#1,	& #2	& #3\\}
}
\def\@juryf{}{}{}
\newcommand{\juryf}[3]{
  \def\@juryf{#1,	& #2	& #3\\}
}
\def\@juryg{}{}{}
\newcommand{\juryg}[3]{
  \def\@juryg{#1,	& #2	& #3\\}
}
\def\@juryh{}{}{}
\newcommand{\juryh}[3]{
  \def\@juryh{#1,	& #2	& #3\\}
}
\def\@juryi{}{}{}
\newcommand{\juryi}[3]{
  \def\@juryi{#1,	& #2	& #3\\}
}
\makeatother

\newcommand\BackgroundPic{%
	\put(0,0){%
		\parbox[b][\paperheight]{\paperwidth}{%
			\includegraphics[height=0.45\paperheight]{bordure.png}%
			\vfill
		}
	}
}
\newcommand\EtiquetteThese{%
	\put(0,0){%
		\parbox[t][\paperheight]{\paperwidth}{%
			\hfill
			\colorbox{blue}{		
				\begin{minipage}[b]{3em}
					\centering\Huge\textcolor{white}{T\\H\\E\\S\\E\\}
					\vspace{0.2cm}
				\end{minipage}
			}
		}
	}
}

\makeatletter
\newcommand{\pagedegarde}{
%\newgeometry{top=2.5cm, bottom=2cm, left=4cm, right=1cm}
%\AddToShipoutPicture*{\BackgroundPic}
%\AddToShipoutPicture*{\EtiquetteThese}
  \begin{titlepage}
\begin{center}
	\begin{minipage}[c]{0.70\linewidth}
		\raggedright \includegraphics[height=1.5cm]{logo_amu}\quad\includegraphics[height=1.5cm]{logo-ISFA}
	\end{minipage}\hfill
	\begin{minipage}[c]{0.30\linewidth}
		\raggedleft \includegraphics[height=1cm]{axa-fund-corpo}
	\end{minipage}\hfill 
\end{center}
\begin{flushleft}
	\vspace{0.2cm}
	\LARGE Institut de Mathématiques de Marseille\\
	\LARGE\textcolor{black!50}{Institut des Sciences Financières et d\rq{}Assurance}\\
	%\vspace{0.2cm}
	\Large ED 184: Ecole doctorale en Mathématiques et Informatique de Marseille\\
	\vspace{0.2cm}

%	LABORATOIRE/UNITE DE RECHERCHE\\
\begin{center}
         \vspace{0.5cm}
        \Huge\color[rgb]{0,0,1} \bfseries Approximations polynomiales de densités de probabilité et applications en assurance
\vspace{0.5cm}
 \end{center}
    \begin{center}
		\LARGE THESE DE DOCTORAT
    \end{center}
	\vspace{0.5cm}
%    Spécialité : indiquer la spécialité s'il y a lieu (intitulés en annexe)\\
    \begin{center}
	\normalsize présentée par\\
        \huge\bfseries Pierre-Olivier Goffard
\vspace{0.5cm}
\end{center}
\begin{center}
	\normalsize pour obtenir le grade de \\
        \Large Docteur de l\rq{}université d\rq{}Aix-Marseille\\
\vspace{0.2cm}
\normalsize Discipline : Mathématiques appliquées

\end{center}
\vspace{1cm}
    \normalsize Soutenue le 29/06/2015 devant le jury :\\
\end{flushleft}
\vspace{0.2cm}
\begin{tabular}{lll}
    \vspace{0.08cm}
	Claude Lefèvre & Professeur à l\rq{}université libre de Bruxelles & Rapporteur \\
    \vspace{0.08cm}
	Partice Bertail &  Professeur à l\rq{}université Paris X  & Rapporteur \\
    \vspace{0.08cm}
	Dominique Henriet & Professeur à l\rq{}école Centrale Marseille & Examinateur \\
    \vspace{0.08cm}
	Xavier Guerrault & Actuaire chez Axa France & Responsable entreprise \\
    \vspace{0.08cm}
	Denys Pommeret & Professeur à l\rq{}université d\rq{}Aix-Marseille & Directeur de thèse \\
\vspace{0.08cm}
	Stéphane Loisel & Professeur à l\rq{}université de Lyon 1 & Co-directeur de thèse \\
\end{tabular}
%  \centering
%      \includegraphics[width=0.4\textwidth]{logoAMU.png}
%      \hfill
%      \includegraphics[width=0.4\textwidth]{axa-fund-corpo.png}\\
%    \vspace{1cm}
%      {\Large ED \no 184 : \'{E}cole doctorale en Mathématiques et Informatique de Marseille}\\
%    \vspace{1cm}
%      {\huge 
%      	{\bfseries Thèse de Doctorat}}\\
%    \vspace{1cm}
%   		{\bfseries pour obtenir le grade de docteur délivré par}\\
%    \vspace{1cm}
%    	{\huge\bfseries \@ecole}\\
%    \vspace{0.5cm}
%    	{\Large{\bfseries Spécialité ``\@specialite''}}\\
%    \vspace{0.5cm}
%    	\textit{présentée et soutenue publiquement par}\\
%    \vspace{0.5cm}
%    	{\Large {\bfseries \@author}} \\
%    \vspace{0.5cm}
%    	le \@date \\
%    \vfill
%       {\LARGE \color[rgb]{0,0,1} \bfseries{\@title}} \\
%    \vfill
%        Directeur de thèse : {\bfseries \@directeur}\\
%        Co-encadrant de thèse : {\bfseries \@encadrant}\\
%	Responsable entreprise :  {\bfseries \@supervisor}\\
%    \vfill
%	\begin{tabular}{>{\bfseries}llr}
%		\large Jury\\
%		\@jurya
%		\@juryb
%		\@juryc
%		\@juryd
%		\@jurye
%		\@juryf
%		\@juryg
%		\@juryh
%		\@juryi
%	\end{tabular}
%	\vfill
%	
%	\textbf{Université d\rq{}Aix-Marseille\\
%	Institut de Mathématiques de Marseille, I2M}\\
%	UMR CNRS 7635, F-06904 Sophia Antipolis, France
  \end{titlepage}




%\restoregeometry  
  
  
}
\makeatother
